%----------------------------------------------------------------------------------------
%	VALUES FOR THE THESIS
%----------------------------------------------------------------------------------------

% \newcommand{\name}{Yu Liu} % Author name
% \newcommand{\thesistitle}{Residential Network Security Study} % Title of the thesis
% \newcommand{\submissiondate}{Auguest, 2020} % Submission date "Month, year"
% \newcommand{\supervisor}{Craig A. Shue} % Supervisor name
% \newcommand{\cosupervisor}{Robert J. Walls} % Co-Supervisor name, comment this line if there is none




%----------------------------------------------------------------------------------------
%	BIBLIOGRAPHY STYLE (pick the style you want)
%----------------------------------------------------------------------------------------

%\usepackage[square, numbers, sort&compress]{natbib} % for bibliography - Square brackets, citing references with numbers, citations sorted by appearance in the text and compressed (as in [4-7])
%\usepackage[longnamesfirst,round]{natbib} % Natural Sciences bibliography

%\bibliographystyle{alpha}
%\bibliographystyle{abbrv}
%\bibliographystyle{Preamble/physics_bibstyle} % You may use a different style adapted to your field
%\bibliographystyle{unsrtnat} % You may use a different style adapted to your field


%----------------------------------------------------------------------------------------
%	YOUR PACKAGES (be careful of package interaction)
%----------------------------------------------------------------------------------------


\usepackage{tikz}
\DeclareRobustCommand{\ballnumber}[1]{\tikz[baseline=(myanchor.base)]
  \node[circle,fill=.,inner sep=1pt] (myanchor) {\color{-.}\sffamily\footnotesize #1};} 
\usepackage{xspace}

\usepackage{url}
\def\UrlBreaks{\do\/\do-}

\usepackage[bookmarks, colorlinks, breaklinks]{hyperref} % Required for links

% Set link colours
\hypersetup{
  linkcolor=MidnightBlue,
  citecolor=MidnightBlue,
  filecolor=black,
  urlcolor=MidnightBlue
}
\newcommand{\MYhref}[3][blue]{\href{#2}{\color{#1}{#3}}}%

%\newcommand{\ballnumbercaption}[1]{\tikz[baseline=(myanchor.base)]
%  \node[circle,fill=.,inner sep=1pt] (myanchor) {\color{-.}\sffamily\footnotesize #1};}

%----------------------------------------------------------------------------------------
%	YOUR DEFINITIONS AND COMMANDS
%----------------------------------------------------------------------------------------

% General
%\newcommand{\thetitle}{Bridging the Gap Between Mobile and Cloud through Fine- and Coarse-grained optimizations.}
\newcommand{\thetitle}{Mobile-Oriented Deep Learning Inference:\\Fine- and Coarse-grained Approaches}
\newcommand{\pdftitle}{\thetitle} 



\newcommand{\todo}[1]{{\color{blue}\textbf{TODO: \textit{#1}}}}
\newcommand{\plan}[1]{{\color{orange}\textbf{Plan: \textit{#1}}}}
\newcommand{\added}[1]{{\color{blue}[[#1]]}}
\newcommand{\eat}[1]{}

\newcommand{\para }[1]{\smallskip \noindent  {\bf \emph{#1}}}
\newcommand{\1}{{\em (i)}}
\newcommand{\2}{{\em (ii)}}
\newcommand{\3}{{\em (iii)}}
\newcommand{\4}{{\em (iv)}}
\newcommand{\5}{{\em (v)}}

%\renewcommand\thesubfigure{(\alph{subfigure})}

% Paper Specific
%\newcommand{\sysname}{\textsc{MD\-Inference}\xspace}%
\newcommand{\modi}{\textit{MODI\xspace}} % or MultiServe
\newcommand{\pieslicer}{\textit{PieSlicer}\xspace}
\newcommand{\mdinference}{\textit{MDInference}\xspace}
\newcommand{\ripcord}{\textit{CremeBrulee}\xspace} % or MultiServe
\newcommand{\thrustthree}{\textit{LayerCake}\xspace} % or MultiServe


% I'm really lazy.
%% Ripcord
\newcommand{\smart}{\emph{BeladyAM}\xspace}
\newcommand{\beladyam}{\emph{BeladyAM-oracle}\xspace}

\newcommand{\x}{$\times$\xspace}

%% MDInference
\newcommand{\framework}{framework\xspace}
\newcommand{\frameworks}{frameworks\xspace}
\newcommand{\accuracy}{aggregate accuracy\xspace}
\newcommand{\Accuracy}{Aggregate accuracy\xspace}

\usepackage{pifont} % http://ctan.org/pkg/pifont
\newcommand{\cmark}{\ding{51}}%
\newcommand{\xmark}{\ding{55}}%

%% Thesis
\newcommand{\subrequests}{subrequests\xspace}
\newcommand{\subrequest}{subrequest\xspace}


\newcounter{challengenum}
\setcounter{challengenum}{0}
\newcommand{\challenge}[1]{\noindent%
   \refstepcounter{challengenum}{\\\large\textbf{Challenge \thechallengenum} : \emph{#1}}
}

\usepackage[super]{nth}
